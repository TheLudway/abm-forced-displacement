\documentclass[12pt]{article}

\usepackage{graphicx}
\usepackage{amsmath}
\usepackage[margin=1in]{geometry}
\usepackage{fancyhdr}
\usepackage{enumerate}
\usepackage[shortlabels]{enumitem}
\usepackage[spanish]{babel}
\usepackage{xurl}
\usepackage{tcolorbox}
\usepackage{titlesec}
\usepackage{listings}
\usepackage{xcolor}
\usepackage{pgfplots}
\usepackage{tikz}
\usepackage{cancel}
\usepackage[hidelinks]{hyperref}
\usepackage{hyperref}

\titleclass{\subsubsubsection}{straight}[\subsection]

\newcounter{subsubsubsection}[subsubsection]
\renewcommand\thesubsubsubsection{\thesubsubsection.\arabic{subsubsubsection}}
\renewcommand\theparagraph{\thesubsubsubsection.\arabic{paragraph}} % optional; useful if paragraphs are to be numbered

\titleformat{\subsubsubsection}
{\normalfont\normalsize\bfseries}{\thesubsubsubsection}{1em}{}
\titlespacing*{\subsubsubsection}
{0pt}{3.25ex plus 1ex minus .2ex}{1.5ex plus .2ex}

\makeatletter
\renewcommand\paragraph{\@startsection{paragraph}{5}{\z@}%
  {3.25ex \@plus1ex \@minus.2ex}%
  {-1em}%
  {\normalfont\normalsize\bfseries}}
\renewcommand\subparagraph{\@startsection{subparagraph}{6}{\parindent}%
  {3.25ex \@plus1ex \@minus .2ex}%
  {-1em}%
  {\normalfont\normalsize\bfseries}}
\def\toclevel@subsubsubsection{4}
\def\toclevel@paragraph{5}
% \def\toclevel@paragraph{6}
\def\toclevel@subparagraph{6}
\def\l@subsubsubsection{\@dottedtocline{4}{7em}{4em}}
\def\l@paragraph{\@dottedtocline{5}{10em}{5em}}
\def\l@subparagraph{\@dottedtocline{6}{14em}{6em}}
\makeatother

\setcounter{secnumdepth}{4}
\setcounter{tocdepth}{4}

% Set up headers and footers
\pagestyle{fancy}
\fancyhf{}  % Clear previous settings

\fancyhead[L]{Alvarado Ludwig}
\fancyhead[C]{Autómatas y Agentes}
\fancyhead[R]{14 de Marzo de 2025}

\fancyfoot[C]{\thepage}
\fancyfoot[C]{\footnotesize Este trabajo está bajo una licencia CC 4.0. Más info: \url{https://creativecommons.org/licenses/by/4.0/}}

\renewcommand{\headrulewidth}{0.2pt}


% Define R Style for listings
\lstdefinestyle{RStyle}{
  language=R,
  basicstyle=\ttfamily\small,
  keywordstyle=\color{blue}\bfseries,
  commentstyle=\color{green!40!black}\itshape,
  stringstyle=\color{red!70!black},
  numbers=left,
  numberstyle=\tiny\color{gray},
  stepnumber=1,
  numbersep=5pt,
  backgroundcolor=\color{gray!10},
  showstringspaces=false,
  breaklines=true,
  frame=single,
  rulecolor=\color{black},
  captionpos=b,
  morekeywords={generator}
}

% Set RStyle as default
\lstset{style=RStyle}

\begin{document}


Se consideran varios \textit{papers} para esta investigación; \textit{Predicting forced displacement using a generalised and automated agent-based simulation}\cite{suleimenova2020predicting}.


En esta investigación consideran 4 aspectos a la hora de modelar desplazamiento forzado; tipo de migrantes, métodos para estructurar los datos disponibles, acercamiento al modelado y medición de la incertidumbre asociada con los datos. Consideran también el movimiento de los IDPs (\textit{Internally displaced people}), incluyendo cuándo deciden irse, cuándo huir y dónde se quedan o qué tanto huyen de su primera elección de destino.

En el \textit{paper} \textit{An agent-based model to identify migration pathways of refugees: the case of Syria}\cite{hebert2017agent} hacen una simulación por Netlogo y adjuntan el diagrama de flujo con el algoritmo que implementaron, a su vez, implementan algo muy interesante, la tolerancia a la violencia y la decisión para irse, incluyendo su religión, étnicidad, dinero, edad, género y estatus familiar, la combinación de estos sirve para determinar la probabilidad de que un agente se marche. Utilizan datos aleatorios debido a la indisponibilidad de los datos. La regla de decisión es un condicional simple, si la violencia del entorno excede a la tolerancia, el agente decide irse, convirtiéndose en un agente migratorio.

Otro \textit{paper} relacionado a la modelación de agentes para la migración es uno hecho por Diego Gutiérrez\cite{Gutierrez2012} en el que realiza una simulación que evidencia cómo la decisión individual de un agente para emigrar, influyen en el comportamiento migratorio colectivo.

Un aspecto importante a analizar en el proyecto son los ingresos de los agentes campesinos, ya que, al verse desplazados, estos no se van a poder acoplar al mercado laboral industrializado de la ciudad. En promedio, los campesinos son menos educados en comparación con el promedio nacional. A nivel nacional la tasa de analfabetismo en 2020 fue de 4.36\%, mientras que la rural es de 18.6\%. Para la sociedad industrial bogotana se requiere de cierto nivel de estudios (la mayoría de ocasiones) para poder obtener un empleo y con ello una fuente de ingresos. La tesis es muy interesante porque presenta muchos factores que le suceden a la población rural al momento de buscar un empleo. También, para el modelo de agentes utiliza una tasa de muerte.

En el artículo \textit{Importancia de la economía campesina en los contextos contemporáneos: una mirada al caso colombiano}\cite{santacoloma2015importancia} la autora expone algunas maneras que los campesinos generan ganancias:

\begin{itemize}
  \item Producción agroecológica.
  \item Intercambio de bienes agrícolas.
  \item Trabajo colectivo.
\end{itemize}

El \textit{paper} \textit{Impacto económico de la violencia armada sobre la producción campesina, caso municipios zona de distensión departamento del Meta, Colombia (1991-2014)}\cite{perez2016impacto} hace una investigación muy buena de cómo el desplazamiento afecta la producción de diferentes productos campesinos. Utiliza regresiones lineales para estos modelos.


Después de haber visto los diferentes \textit{papers} se obtienen las siguientes ideas para el modelo del proyecto:

\begin{itemize}
  \item Tener un valor de violencia percibida.
  \item Tener dos agentes: campesinos y grupo armado.
  \item Los campesinos pueden tener dos estados: \texttt{No desplazado} y \texttt{Desplazado}.
  \item Se contempla una muerte de los agentes para controlar la sobre población.
  \item Los agentes campesinos tendrán un atributo de tolerancia al peligro que se ve afectada por su entorno (si varios campesinos se van esta baja).
  \item La decisión de migrar se toma cuando la percepción de peligro es mayor a la tolerancia.
  \item Los agentes campesinos tienen métodos para generar ingresos como exportación agrícola.
\end{itemize}

Entonces, estamos en que tales dijo este autor.


\bibliographystyle{ieeetr}
\bibliography{referencias}





\end{document}
