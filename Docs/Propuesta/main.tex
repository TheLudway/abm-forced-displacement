\documentclass{article}

\usepackage{amsmath, amsthm, amssymb, amsfonts}
\usepackage{thmtools}
\usepackage{graphicx}
\usepackage{setspace}
\usepackage{geometry}
\usepackage{float}
\usepackage[hidelinks]{hyperref}
\usepackage[utf8]{inputenc}
\usepackage[spanish,es-nodecimaldot]{babel}
\usepackage{framed}
\usepackage[dvipsnames]{xcolor}
\usepackage{tcolorbox}
\usepackage{tikz}
\usepackage{caption}
\usepackage{longtable}
\usepackage{pdflscape}
\usepackage{svg}
\usepackage{subcaption}
\usepackage{caption}
\usepackage{multirow}
\usepackage{array}
\usepackage{listings}
\usepackage{cancel}

\colorlet{LightGray}{White!90!Periwinkle}
\colorlet{LightOrange}{Orange!15}
\colorlet{LightGreen}{Green!15}



\newcommand{\HRule}[1]{\rule{\linewidth}{#1}}

\declaretheoremstyle[name=Theorem,]{thmsty}
\declaretheorem[style=thmsty,numberwithin=section]{theorem}
\tcolorboxenvironment{theorem}{colback=LightGray}

\declaretheoremstyle[name=Proposition,]{prosty}
\declaretheorem[style=prosty,numberlike=theorem]{proposition}
\tcolorboxenvironment{proposition}{colback=LightOrange}

\declaretheoremstyle[name=Principle,]{prcpsty}
\declaretheorem[style=prcpsty,numberlike=theorem]{principle}
\tcolorboxenvironment{principle}{colback=LightGreen}

\newcolumntype{L}[1]{>{\raggedleft\let\newline\\\arraybackslash\hspace{0pt}}m{#1}}
\newcolumntype{C}[1]{>{\centering\let\newline\\\arraybackslash\hspace{0pt}}m{#1}}
\newcolumntype{R}[1]{>{\raggedright\let\newline\\\arraybackslash\hspace{0pt}}m{#1}}

\setstretch{1.2}
\geometry{
    textheight=9in,
    textwidth=5.5in,
    top=1in,
    headheight=12pt,
    headsep=25pt,
    footskip=30pt
}

\lstdefinestyle{bashstyle}{
    language=bash,
    basicstyle=\ttfamily,
    backgroundcolor=\color{gray!10},
    keywordstyle=\color{blue},
    commentstyle=\color{green!40!black},
    stringstyle=\color{red},
    showstringspaces=false,
    numbers=left,
    numberstyle=\tiny\color{gray},
    breaklines=true,
    breakatwhitespace=true,
    frame=tb,
    rulecolor=\color{black!70},
    framerule=0.5pt,
    tabsize=4,
    captionpos=b
}

\lstdefinestyle{javastyle}{
    language=Java,
    basicstyle=\ttfamily,
    backgroundcolor=\color{gray!10},
    keywordstyle=\color{blue},
    commentstyle=\color{green!40!black},
    stringstyle=\color{red},
    showstringspaces=false,
    numbers=left,
    numberstyle=\tiny\color{gray},
    breaklines=true,
    breakatwhitespace=true,
    frame=tb,
    rulecolor=\color{black!70},
    framerule=0.5pt,
    tabsize=4,
    captionpos=b
}

% ------------------------------------------------------------------------------

\begin{document}

% ------------------------------------------------------------------------------
% Cover Page and ToC
% ------------------------------------------------------------------------------

\title{ \normalsize \textsc{}
	\\ [2.0cm]
	\HRule{1.5pt} \\
	\LARGE \textbf{\uppercase{Cambio en los Ingresos de Familias Desplazadas del Meta por el Conflicto Armado}
		\HRule{2.0pt} \\ [0.6cm] \LARGE{Universidad de Bogotá Jorge Tadeo Lozano} \vspace*{10\baselineskip}}
}
\date{}
\author{\textbf{Alvarado Becerra Ludwig} \\
  \textbf{Vera Soto Julián David} \\
	Simulación Estocástica - 20251S}

\maketitle
\thispagestyle{empty}
\newpage

\thispagestyle{empty}
\newpage
\setcounter{page}{1}

Este proyecto es el que se está realizando para la materia de \textit{Autómatas y Agentes} que está cursando uno de los autores, Ludwig Alvarado. Por lo tanto, mucho de la información presente en este documento, ya estaba definida antes de la asignación del presente trabajo.


\section{Descripción del problema}

La migración forzada es un fenómeno que se empeoró a partir de la década de 1990, haciendo que la población afectada se desplace a otras zonas geográficas, generalmente en ciudades urbanas muy diferentes a lo que llevan haciendo toda su vida. Este movimiento incrementa la condiciones de pobreza de los afectados \cite{ruiz2011desplazamiento}. El principal grupo afectado es la población campesina, quienes manejan un modelo de producción agrícola en sus tierras y cuando son forzados a ir a una zona urbana, no pueden adaptarse del todo al modelo agroindustrial que rigen las grandes ciudades del país\cite{bello2003desplazamiento}. Diferentes factores afectan a la economía de los desplazados, sin embargo, no existe una literatura clara para medir el cambio de los ingresos.


\section{Contexto}

La mayoría de campesinos que abandonan sus hogares es por problemas asociados con el conflicto armado. En 1997 el Estado Colombiano reconoce el éxodo forzado como una problemática que exige acciones de política pública enfocadas en la necesidad de prevenir el fenómeno\cite{villa2006desplazamiento}. A pesar de que ya hay leyes para la atención de la población desplazada, los números no han mostrado muchos resultados. A día de hoy hay $9.882.219$ víctimas identificadas\cite{unidadvictimas_ruv_2025}, corresponde cerca del $16,83 \%$ de la población total del país\cite{eswiki:165709323} y supera la población estimada de Bogotá en 2023\cite{eswiki:165573820}. A partir del año 1985 el departamento del Meta ha tenido un incremento de víctimas por el desplazamiento forzado debido a la presencia de diferentes grupos armados\cite{solano2020determinantes}.


\section{Estado del arte}

\section{Objetivo general}

\section{Metodología}


\bibliographystyle{ieeetr}
\bibliography{referencias}





\end{document}
